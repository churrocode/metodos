\section{Conclusiones}

Fijamos una tolerancia para cada método de acuerdo a las observaciones que pudimos hacer en base a los resultados: dicha tolerancia fue de ${10^-9}$ para
todos los métodos menos para bisección, ${10^-15}$ para esta última que es el $\epsilon$ que tomamos para evaluar cuándo dos número de punto flotante son iguales. A partir de dichos valores, la precisión de cada método se mantiene constante con una cota de error por debajo de ${10^-12}$.

En cuanto al tiempo de ejecución no pudimos observar grandes cambios de método a método, pero sin embargo es sencillo deducir las diferencias de tiempos basándonos en la operaciones que se computan en cada método. Para el primer método, el cuál calcula el inverso multiplicativo del cero de $f(x) = x^2 - \alpha$




En vista de los resultados, concluimos que el método de la bisección es que peor resultados nos trajo en relación a todos los parámetros que evaluamos (tolerancia de error, tiempo de ejecución y cantidad de iteraciones).